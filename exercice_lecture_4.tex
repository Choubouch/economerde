\documentclass{article}
\usepackage[utf8]{inputenc}
\usepackage{array}
\usepackage{titling} % Pour des commandes de titre avancées
\usepackage[left=3cm,right=3cm,top=3cm,bottom=3cm]{geometry}
\usepackage{amsmath}
\usepackage{graphicx}
% Définition du titre, des auteurs et de la date

\title{\Huge \textbf{Exercice lecture n°4}}
\author{Matti Comba \and Mathis Verdan}
\date{\today}
\begin{document}

% Commande pour le titre avec séparation horizontale
\maketitle
\begin{center}
  \rule{\linewidth}{0.5mm}
\end{center}


% Contenu du document
\section{Mesures de $r$, $g_Y$ et $B$}


\begin{figure}[h!]
  \centering
  \begin{minipage}{0.8\textwidth}
      \centering
      \includegraphics[width=\textwidth]{"croissance.png"}
      \caption{Données d'Eurostat sur la croissance $g_y$}
  \end{minipage}
\end{figure}

\begin{figure}[h!]
  \centering
  \begin{minipage}{0.8\textwidth}
      \centering
      \includegraphics[width=\textwidth]{"dette_publique.png"}
      \caption{Données d'Eurostat sur la dette publique $B$}
  \end{minipage}
\end{figure}

\begin{figure}[h!]
    \centering
    \begin{tabular}{ | c | c | c |  c |  c |  c |  c |  c | }
      \hline
      Pays & Colombie & Hongrie & Italie & USA  & Norvège & Allemagne & France \\ \hline
      Taux d'intérêt (en \% par an 2023 ) & 11,3 & 7,5 & 4,3 & 4,0 & 3,4 & 2,4 & 3,0 \\ \hline
    \end{tabular}
    \caption{Taux d'intérêts à long terme $r$ en \% par an (source OCDE)}
\end{figure}
    



\section{D’après les données actuelles, pensez-vous que la dette publique en France (ou dans un autre pays
de votre choix) est soutenable ?
}

\section{Comment pensez-vous que les variables clés évolueront ? Qu’est-ce que cela implique pour la
soutenabilité de la dette ?}

\end{document}
