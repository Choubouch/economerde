\documentclass{article}
\usepackage[utf8]{inputenc}
\usepackage{array}
\usepackage{titling} % Pour des commandes de titre avancées
\usepackage[left=3cm,right=3cm,top=3cm,bottom=3cm]{geometry}
\usepackage{amsmath}
% Définition du titre, des auteurs et de la date

\title{\Huge \textbf{Exercice lecture n°2}}
\author{Matti Comba \and Mathis Verdan}
\date{\today}

\begin{document}

% Commande pour le titre avec séparation horizontale
\maketitle
\begin{center}
  \rule{\linewidth}{0.5mm}
\end{center}


% Contenu du document
\section{Impact des Jeux Olympiques et de l'Euro 2016 en France}
Pour comparer les 1,22 milliards d'euros obtenus au cours de l'Euro 2016 et les 3,8 milliards que pourraient rapporter les Jeux Olympiques 2024, on prend en compte l'inflation pour ramener les prix à la même année. Pour cela, on se sert des données sur l'inflation obtenues sur Internet : 

\begin{center}
    \begin{tabular}{ | c | c | c |  c |  c |  c |  c |  c |  c | }
      \hline
      Année & 2016 & 2017 & 2018 & 2019 & 2020 & 2021 & 2022 & 2023 \\ \hline
      Inflation (en \% ) & 0,2 & 1 & 1,8 & 1,1 & 0,5 & 1,6 & 5,2 & 4,9 \\ \hline
    \end{tabular}
\end{center}

Cela signifie que les 1,22 milliards d'euros de 2016 valent :
$$ 1,22 \times 1,002 \times 1,01 \times 1,018 \times 1,011 \times 1,005 \times 1,016 \times 1,052 \times 1,049 = 1,43 \text{ milliards d'euros en 2023}$$
On constate donc que même en prenant l'inflation en compte, les JO seraient un événement presque 3 fois plus important pour l'économie que l'Euro. Toutefois, il ne faut pas oublier que les 3,8 milliards d'euros sont une estimation et que nous pourrons faire une comparaison précise seulement à la fin des JO.

\section{Impact des touristes américains sur l'économie française}
  
    Selon cette étude, le facteur multiplicateur retenu est $ C_{mult} = 1.25$. 
  
  L'annexe 9 indique de plus qu'un touriste américain reste en moyenne $N_{jours} = 11.06$ jours et dépense en moyenne $D_{moy}=224.12 $ \texteuro hors trajet (que nous ne prendrons pas en compte car les compagnies aériennes choisies ne sont pas forcément françaises et donc ces dépenses ne sont pas injectées dans l'économie locale).

  L'inflation en cumulée $I_{cumulee}$ entre 2016 et 2023 étant de 17.47\% , on obtient yn impact économique équivalent pour les JO de la venue de $N_{americains} = 200 000$ américains :
  $$ Impact = C_{mult} \times D_{moy} \times N_{jours} \times N_{americains} \times I_{cumulee}$$

  Soit un impact de 728 millions d'euros.
\section{Conclusion : impact des JO sur l'économie locale}



  Afin d'estimer l'impact de cette rentrée d'argent dans l'économie parisienne, nous avons évalué le PIB de l'Ile-de-France en 2023. 

  
  \vspace{0.5cm}
  \begin{center}
    \begin{tabular}{ | c | c | c |  c |  c |  c |  c |  c |  c | }
      \hline
      Année & 2021 & 2022 & 2023  \\ \hline
      Croissance (en \% ) & 6,4 & 2,5 & 1,1  \\ \hline
    \end{tabular}
  \end{center}
    \vspace{0.5cm}

  En 2020, ce PIB s'élevait à 764,8 milliards d'euros. En prenant compte de la croissance française présentée ci-dessus, nous obtenons un PIB actualisé de 843,3 milliards d'euros.

  On constate donc que l'impact économique des JO sur l'Ile-de-France ne représente que 0.45\% de son PIB.

  Dans un premier temps on peut dire que ce n'est qu'une infime part du PIB. Néanmoins, il s'agit d'un évènement qui ne dure que deux semaines et  dans une période de faible croissance et de forte inflation cela peut contribuer à booster l'économie francilienne à court terme. 


  De plus, l'acquisition d'infrastructures sportives via des investissement publics notamment non prise en compte dans l'étude d'impact économique pourra à moyen terme permettre l'organisation plus aisée d'évènements sportifs de ce genre.



\end{document}
