\documentclass{article}
\usepackage[utf8]{inputenc}
\usepackage{array}
\usepackage{titling} % Pour des commandes de titre avancées
\usepackage[left=3cm,right=3cm,top=3cm,bottom=3cm]{geometry}
\usepackage{amsmath}
\usepackage{graphicx}
\usepackage{float}
% Définition du titre, des auteurs et de la date

\title{\Huge \textbf{Exercice lecture n°3}}
\author{Matti Comba \and Mathis Verdan}
\date{\today}
\begin{document}

% Commande pour le titre avec séparation horizontale
\maketitle
\begin{center}
  \rule{\linewidth}{0.5mm}
\end{center}


% Contenu du document
\section{Mesures de $m$ et $z$}

\begin{figure}[H]
  \centering
  \begin{minipage}{0.8\textwidth}
      \centering
      \includegraphics[width=\textwidth]{"smic.png"}
      \caption{Données d'Eurostat sur les salaires minimaux}
  \end{minipage}
\end{figure}

  \begin{figure}[H]
    \centering
    \begin{minipage}{0.8\textwidth}
        \centering
        \includegraphics[width=\textwidth]{"allocations.png"}
        \caption{Données d'Eurostat sur les salaires minimaux}
    \end{minipage}
  \end{figure}

\begin{figure}[H]
  \centering
  \begin{minipage}{0.8\textwidth}
      \centering
      \includegraphics[width=\textwidth]{"prix_elec.png"}
      \caption{Données d'Eurostat sur les prix de l'électricité dans l'industrie}
  \end{minipage}
\end{figure}

\begin{figure}[H]
  \centering
  \begin{minipage}{0.8\textwidth}
      \centering
      \includegraphics[width=\textwidth]{"prix_prod.png"}
      \caption{Données d'Eurostat sur les coûts de production dans l'industrie}
  \end{minipage}
\end{figure}



\clearpage

\section{Taux de chômage entre pays et au fil du temps}
\begin{figure}[H]
  \centering
  \begin{minipage}{0.8\textwidth}
      \centering
      \includegraphics[width=\textwidth]{"chomage.png"}
      \caption{Données d'Eurostat sur le chomage}
  \end{minipage}
\end{figure}

Si l'on compare les différents pays entre eux, on voit que la Grèce se démarque avec un taux de chômage bien plus élevé que dans les autres pays.  L'Italie et la France ont également des taux de chômage assez élevés (avec 7,7\% et 7,3\% de chômage pour ces deux pays en 2023). À l'opposé de cela, l'Allemagne a un taux de chômage assez faible durant toute la période temporelle, et la Pologne voit sont taux de chômage diminuer pour rejoindre l'Allemagne depuis 2019.

On constate également que les prix de l'électricité pour la Grèce sont assez élevés tout au long de la série temporelle. L'Italie et la Roumanie se distinguent aussi par leur prix assez supérieurs aux autres pays. (pour la Roumanie, les prix ont surtout beaucoup augmenté depuis 2021). En Pologne, France et Allemagne, les prix sont relativement bas.
\section{comparaison avec le modèle}

\end{document}
