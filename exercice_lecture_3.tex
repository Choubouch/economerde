\documentclass{article}
\usepackage[utf8]{inputenc}
\usepackage{array}
\usepackage{titling} % Pour des commandes de titre avancées
\usepackage[left=3cm,right=3cm,top=3cm,bottom=3cm]{geometry}
\usepackage{amsmath}
\usepackage{graphicx}
\usepackage{float}
% Définition du titre, des auteurs et de la date

\title{\Huge \textbf{Exercice lecture n°3}}
\author{Matti Comba \and Mathis Verdan}
\date{\today}
\begin{document}

% Commande pour le titre avec séparation horizontale
\maketitle
\begin{center}
  \rule{\linewidth}{0.5mm}
\end{center}


% Contenu du document
\section{Mesures de $m$ et $z$}

\subsection{Mesures de $z$}

\begin{figure}[H]
  \centering
  \begin{minipage}{0.8\textwidth}
      \centering
      \includegraphics[width=\textwidth]{"smic.png"}
      \caption{Données d'Eurostat sur les salaires minimaux}
  \end{minipage}
\end{figure}
\vspace{0.5cm}

Afin de mesurer l'indicateur z, nous avons utilisé principalement le niveau de salaire minimal dans les différents pays. En effet, cette contrainte pour les entreprises est universelle et donne une bonne idée de la protection des employés.

On constate sur ce graphique que deux groupes de pays se distinguent : d'un côté un bloc plus Europe de l'Ouest avec la France et l'Allemagne qui ont un fort salaire minimal. De l'autre, on observe des pays de l'Est comme la Pologne ou la Roumanie qui sont un cran en dessous en terme de salaire minimum.


  \begin{figure}[H]
    \centering
    \begin{minipage}{0.8\textwidth}
        \centering
        \includegraphics[width=\textwidth]{"allocations.png"}
        \caption{Données d'Eurostat sur la part de chômeurs inscrits percevant de l'aide de l'Etat}
    \end{minipage}
  \end{figure}
\vspace{0.5cm}

Le second indicateur choisi pour z est celui de la part de chômeurs inscrits percevant de l'aide de l'Etat. Cet indicateur permet de saisir à quel point l'Etat est protecteur ou non et permet de donc de donner un ordre d'idée de z. Néanmoins, le fait qu'il soit en pourcentage et pas en valeur absolue (ie en hauteur des allocations versées) rend cette mesure moins fiable.


\subsection{Mesures de $m$}


\begin{figure}[H]
  \centering
  \begin{minipage}{0.8\textwidth}
      \centering
      \includegraphics[width=\textwidth]{"prix_elec_2.png"}
      \caption{Données d'Eurostat sur les prix de l'électricité dans l'industrie}
  \end{minipage}
\end{figure}
\vspace{0.5cm}

Afin de mesurer le markup $m$, nous avons choisi plusieurs indicateurs. Le premier est celui du prix de l'énergie. En effet, cela influence énormément la capacité d'une entreprise à dégager une marge et à être compétitive.

Ce graphique ne permet cependant pas de dégager de réal gap de prix sur le long terme entre les pays considérés (en effet les prix ont augmenté dernièrement notamment en France mais de manière transitoire en raison de maintenance du parc nucléaire).

\begin{figure}[H]
  \centering
  \begin{minipage}{0.8\textwidth}
      \centering
      \includegraphics[width=\textwidth]{"prix_prod.png"}
      \caption{Données d'Eurostat sur les coûts de production dans l'industrie}
  \end{minipage}
\end{figure}
\vspace{0.5cm}

Ce dernier graphique parle de lui-même en terme de capacité des entreprises à gérer leurs marges.

On constate une tendance similaire à celle du début : Europe de l'Ouest a des coûts plus faibles, Europe de l'Est plus élevés.



\clearpage

\section{Taux de chômage entre pays et au fil du temps}
\begin{figure}[H]
  \centering
  \begin{minipage}{0.8\textwidth}
      \centering
      \includegraphics[width=\textwidth]{"chomage.png"}
      \caption{Données d'Eurostat sur le chomage}
  \end{minipage}
\end{figure}

Si l'on compare les différents pays entre eux, on voit que la Grèce se démarque avec un taux de chômage bien plus élevé que dans les autres pays.  L'Italie et la France ont également des taux de chômage assez élevés (avec 7,7\% et 7,3\% de chômage pour ces deux pays en 2023). À l'opposé de cela, l'Allemagne a un taux de chômage assez faible durant toute la période temporelle, et la Pologne voit sont taux de chômage diminuer pour rejoindre l'Allemagne depuis 2019.

On constate également que les prix de l'électricité pour la Grèce sont assez élevés tout au long de la série temporelle. L'Italie et la Roumanie se distinguent aussi par leur prix assez supérieurs aux autres pays. (pour la Roumanie, les prix ont surtout beaucoup augmenté depuis 2021). En Pologne, France et Allemagne, les prix sont relativement bas.
\section{comparaison avec le modèle}

\end{document}
